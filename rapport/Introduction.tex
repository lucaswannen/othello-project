\section{Introduction}
Dans le cadre de notre EC d'Algorithmique avancée et programmation C, nous avons du concevoir une version simplifiée du jeu Othello. Notre groupe était composé de quatre membres : Bianca, Lucas, Sipeng et Mathias.\\

La méthodologie de travail était imposée à suivre le modèle de Cycle en V : \\
\begin{itemize}
\item Analyse : spécification des TAD et Analyse Descendante
\item Conception préliminaire
\item Conception détaillée
\item Développement
\item Tests Unitaires
\end{itemize}\\

Ainsi, le planning de travail a été conforme à cette méthodologie. Pourtant, il a fallu plusieurs fois revenir aux phases précédentes pour corriger les différentes erreurs de cohérence (signatures des fonctions, dépéndance entre elles voir décomposition des fonctions trop complexes), pour rajouter d'autres opérations aux TADs nécessaires pour le développement. En effet, notre analyse descendante a évolué tout au long de notre projet. \\

Notre version d'othello devait comporter deux modes de jeu: l'option \textit{standard} (permettant de faire une partie contre une intelligence artificielle) et l'option \textit{tournoi} (versionpermettant à deux IA de jouer l'une contre l'autre). \\

L’interface homme machine est libre (mode texte simple, utilisationde la librairiencurses, utilisation d’une librairie graphique comme GTK, etc.).\\

\newpage

\section{Conclusion}

Ce projet nous a permis de mettre en application les notions théoriques que nous avons vu au cours du semestre. De plus, nous avons pu grace à lui nous familiariser avec l'outil de travil qu'est Git, que nous serons amenés à réutiliser à de nombreuses reprises au cours des prochaines années, et même dans notre vie professionnelle. En outre, la coordination entre les membres de l'équipe et la répartition du travail ne devaient pas être négligées. En effet, bien que nous ayons déjà été confrontés à ces problématiques de nombreuses fois au cours de notre scolarité, cette dernière prenait avec ce projet une tout autre dimension : l'utilisation de Git ou encore la rigueur qu'impose l'implémentation de fonctions, notamment dans le nom ou le type des variables, étaient pour nous de nouveaux enjeux. Ce fut donc une expérience enrichissante qui nous a permis de mieux situer les complexités inhérentes à la gestion d'un projet de groupe, dans le domaine de l'informatique plus particulièrement.\\

\textbf{Remarques personnelles}\\

\textit{Mathias Van Audenhove}\\

Personnellement, ce projet m'a apporté beaucoup d’expérience. J'avais certainement déjà les connaissances nécessaires à la conception logique de ce projet, mais je subissais encore des lacunes dans le domaine de l'implémentation. Ces dernières ont bien vite disparu. De plus, l'utilisation de Git et l'organisation d'un projet de groupe de cette envergure étaient également pour moi de nouveaux enjeux, avec lesquels j'ai pu me familiariser. Je le vois bien aujourd'hui : Au début de l'année, je n'avais pas la moindre idée concrète de la manière dont pouvait être géré un projet de ce genre. Aujourd'hui, et grâce à l'expérience acquise au cours de ce semestre, j'aborderai mes futurs projets confiant et en sachant quelles directions prendre pour les mener à bien.\\

\textit{Lucas Wannenmacher}\\

J'ai trouvé le projet très formateur, car il m'a permis non seulement de me perfectionner en C mais aussi d'apprendre à utiliser des outils tels que Git ou Doxygen (outils que j'ai d'ailleurs trouvé très pratiques) . Le projet a parfois demandé une une charge de travail relativement importante, notamment pour résoudre des bugs dont nous avions pris du temps à trouver l'origine. Le projet a aussi à mes yeux mis en évidence l'importance d'écrire ses fonctions et tests avec une certaine rigueur.\\

\textit{Sipeng Zheng}\\

Grâce à ce projet, j'ai pu améliorer mon travail en équipe, un aspect qui n'est pas aussi simple et évident que d'écrire du code de son côté. Chacun d'entre nous avait des idées différentes et le fait de se mettre d'accord sur les différents points, afin de bien se synchroniser pour le but commun du projet, peut être assez compliqué. De plus, l'outil Git nous a beaucoup facilité notre travail. Cela nous a permis de voir également le progrès des autres et de mieux collaborer.\\

\textit{Maria-Bianca Zugravu}\\

Grâce à ce projet, j'ai eu l'occasion de mettre en oeuvre les connaissances théoriques acquises dans cet EC et d'enrichir ces compétences. De même, en tant que chef de projet, j'ai découvert les différentes problématiques liées aux aspects organisationnels, tels que : la mise en oeuvre de la gestion de projet (le Git) afin de suivre en continu l'évolution et de prévoir les tâches à effectuer. Pourtant, ceci m'a posé des problèmes, le fait de synchroniser tout le monde et de s'assurer que chacun fait son travail n'a pas été tout à fait facile. En essayant de rester compréhensible et de maintenir une bonne ambiance au sein de l'équipe j'ai du accepter l'implication oscillante et parfois inexistante. Malgré tout cela, j'ai pu travailler en plus ce qui m'a permis d'apprendre beaucoup plus. 

\section{Bilan}

\begin{table}[h]
\centering
\begin{tabular}{|c|c|}
  \hline
Points accomplis                                                              & Améliorations                                                                                                                                                  \\ \hline
Première version fonctionnelle en mode standard/tournoi                                                              & Optimiser et passer à des versions plus stables                                                                                                                                                  \\ \hline
Affichage cohèrent et illustratif                                                              & \begin{tabular}[c]{@{}l@{}}Affichage graphique + intéractif (mode standard : \\indiquer les possibilités + nb des pions retournés)\end{tabular}                                                                                                                                                  \\ \hline
IA qui marche en utilisant alpha\-beta                                                              & \begin{tabular}[c]{@{}l@{}}Utiliser des algorithmes plus puissants\\($-$ de lenteur et taux de succès $+$ élevé) \end{tabular}                                                                                                                                                 \\ \hline
\begin{tabular}[c]{@{}l@{}}Utiliser au maximum les opérations des TAD\\ afin d'éviter de créer des nouvelles fonctions (Mise à jour)\end{tabular}                                                                     & Problème de portabilité     \\ \hline 
Tests unitaires pour chaque fonction (prèsque)                                                              & $+$ de tests afin de découvrir les bugs en amont                                                                                                                                                  \\ \hline
                                                              & $+$ plus de séances en équipe pour assurer l'implication                                                                                                                                                  \\ \hline
\end{tabular}
\end{table}


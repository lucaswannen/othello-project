% ----------------------------------------------- TAD Couleur --------------------------------------------

\begin{tad}
\tadNom{Couleur}
\tadParametres{}
\tadDependances{}
\begin{tadOperations}{Couleur}
    \tadOperation{blanc}{}{\tadUnParam{Couleur}}
    \tadOperation{noir}{}{\tadUnParam{Couleur}}
    \tadOperation{changerCouleur}{\tadUnParam{Couleur}{}}{\tadUnParam{Couleur}}
\end{tadOperations}
\begin{tadSemantiques}{Couleur}
\tadSemantique{changerCouleur}{L’opération qui permet de changer la couleur}
\end{tadSemantiques}
\begin{tadAxiomes}
    \tadAxiome{changerCouleur(blanc) = noir}
    \tadAxiome{changerCouleur(noir) = blanc}
\end{tadAxiomes}
\end{tad} 
\begin{center}
 \HRule \\[0.4cm]   
\end{center}

% ------------------------------------------------- TAD PION -----------------------------------------------
\begin{tad}
\tadNom{Pion}
\tadParametres{}
\tadDependances{Couleur}
\begin{tadOperations}{Pion}
    \tadOperation{pion}{\tadUnParam{Couleur}{}}{\tadUnParam{Pion}}
    \tadOperation{obtenirCouleur}{\tadUnParam{Pion}{}}{\tadUnParam{Couleur}}
    \tadOperation{retournerPion}{\tadUnParam{Pion}{}}{\tadUnParam{Pion}}
\end{tadOperations}
\begin{tadSemantiques}{Couleur}
    \tadSemantique{obtenirCouleur}{L’opération qui permet d'obtenir la couleur d'un pion.}
    \tadSemantique{retournerPion}{L’opération qui permet de retourner un pion de l'adversaire, c'est à dire l'encadrer par deux pions à nous et de changer sa couleur. }
\end{tadSemantiques}
\begin{tadAxiomes}
    \tadAxiome{obtenirCouleur(pion(c)) = c}
    \tadAxiome{retournerPion(retournerPion(p)) = p}
\end{tadAxiomes}
\end{tad}
\begin{center}
 \HRule \\[0.4cm]   
\end{center}
% --------------------------------------------- TAD POSITION ------------------------------------------------
\begin{tad}
\tadNom{Position}
\tadParametres{}
\tadDependances{[\![1..8]\!], [\![a..h]\!]}
\begin{tadOperations}{Position}
    \tadOperation{position}{\tadDeuxParams{[\![1..8]\!]}{[\![a..h]\!]}}{\tadUnParam{Position}}
    \tadOperation{obtenirAbcisse}{\tadUnParam{Position}{}}{\tadUnParam{[\![1..8]\!]}}
    \tadOperation{obtenirOrdonnée}{\tadUnParam{Position}{}}{\tadUnParam{[\![a..h]\!]}}
\end{tadOperations}
\begin{tadSemantiques}{Position}
    \tadSemantique{obtenirAbcisse}{L’opération qui permet d'obtenir l'abcisse d'une position, c'est à dire la ligne sur laquelle on se positionne.}
    \tadSemantique{obtenirOrdonnée}{L’opération qui permet d'obtenir l'ordonnée d'une position, c'est à dire la colonne sur laquelle on se positionne.}
\end{tadSemantiques}
\begin{tadAxiomes}
    \tadAxiome{obtenirAbcisse(position(i,j)) = i où i $\in$ [\![1..8]\!]}
    \tadAxiome{obtenirOrdonnée(position(i,j)) = j où j $\in$ [\![a..h]\!]}
\end{tadAxiomes}
\end{tad}
\newpage
% -------------------------------------------- TAD PLATEAU ---------------------------------------------
\begin{tad}
\tadNom{Plateau}
\tadParametres{}
\tadDependances{Pion, Position, Booléen}
\begin{tadOperations}{Plateau}
    \tadOperation{plateau}{}{\tadUnParam{Plateau}}
    \tadOperationAvecPreconditions{poserPion}{\tadTroisParams{Pion}{Position}{Plateau}}{\tadUnParam{Plateau}}
    \tadOperationAvecPreconditions{enleverPion}{\tadDeuxParams{Position}{Plateau}}{\tadUnParam{Plateau}}
    \tadOperationAvecPreconditions{retournerPion}{\tadDeuxParams{Position}{Plateau}}{\tadUnParam{Plateau}}
    \tadOperationAvecPreconditions{obtenirPion}{\tadDeuxParams{Position}{Plateau}}{\tadUnParam{Pion}}
    \tadOperation{caseEstVide}{\tadDeuxParams{Position}{Plateau}}{\tadUnParam{Booléen}}
\end{tadOperations}
\begin{tadSemantiques}{Couleur}
    \tadSemantique{poserPion}{L’opération qui permet de poser un pion sur un plateau.}
    \tadSemantique{enleverPion}{L’opération qui permet d'enlever un pion depuis un plateau.}
    \tadSemantique{retournerPion}{L’opération qui permet de retourner un pion sur un plateau à partir de sa position.}
    \tadSemantique{obtenirPion}{L’opération qui permet d'obtenir un pion en fonction de sa position sur un plateau.}
    \tadSemantique{caseEstVide}{L’opération qui permet de vérifier si la case définie par une position sur le plateau est vide.}
\end{tadSemantiques}
\begin{tadPreconditions}{Plateau}
    \tadPrecondition{poserPion(p,pos,plateau)} \textit{caseEstVide(pos,plateau)}
    \tadPrecondition{enleverPion(pos,plateau)}  \textit{non(caseEstVide(pos,plateau))}
    \tadPrecondition{retournerPion(pos,plateau)} \textit{non(caseEstVide(pos,plateau))}
    \tadPrecondition{obtenirPion(pos,plateau)} \textit{non(caseEstVide(pos,plateau))}
\end{tadPreconditions}
\begin{tadAxiomes}
    \tadAxiome{obtenirPion(pos,poserPion(p,pos,plateau)) = p}
    \tadAxiome{retournerPion(pos,retournerPion(pos,plateau)) = plateau}
    \tadAxiome{caseEstVide(pos,plateau())}
    \tadAxiome{caseEstVide(pos,enleverPion(pos,plateau))}
    \tadAxiome{non caseEstVide(pos,poserPion(p,pos,plateau))}
\end{tadAxiomes}
\end{tad}
\begin{center}
 \HRule \\[0.4cm]   
\end{center}
% --------------------------------------------- TAD COUP -------------------------------------------------
\begin{tad}
\tadNom{Coup}
\tadParametres{}
\tadDependances{Pion, Position, Booléen}
\begin{tadOperations}{Coup}
    \tadOperation{coup}{\tadDeuxParams{Pion}{Position}}{\tadUnParam{Coup}}
    \tadOperation{obtenirPion}{\tadUnParam{Coup}{}}{\tadUnParam{Pion}}
    \tadOperation{obtenirPosition}{\tadUnParam{Coup}{}}{\tadUnParam{Position}}
    \tadOperation{estValide}{\tadUnParam{Coup}{}}{\tadUnParam{Booléen}}
\end{tadOperations}
\begin{tadSemantiques}{Coup}
    \tadSemantique{obtenirPion}{L’opération qui permet d'obtenir le pion correspondant à un coup.}
    \tadSemantique{obtenirPosition}{L’opération qui permet d'obtenir la position correspondante à un coup.}
    \tadSemantique{estValide}{L’opération qui permet de vérifier si un coup est valide, c'est à dire s'il n'est pas en dehors des limites du plateau.}
\end{tadSemantiques}
\begin{tadAxiomes}
    \tadAxiome{obtenirPion(coup(p,pos)) = p}
    \tadAxiome{obtenirPosition(coup(p,pos)) = pos}
\end{tadAxiomes}
\end{tad}

\newpage
% -------------------------------------- TAD COUPS -----------------------------------------------------------
\begin{tad}
\tadNom{Coups}
\tadParametres{Coup}
\tadDependances{Coup, Booléen, Naturel, NaturelNonNul}
\begin{tadOperations}{Coups}
    \tadOperation{coups}{}{\tadUnParam{Coups}}
    \tadOperation{estCoupValide}{\tadDeuxParams{Coups}{Coup}}{\tadUnParam{Booléen}}
    \tadOperationAvecPreconditions{iemeCoup}{\tadDeuxParams{Coups}{NaturelNonNul}}{\tadUnParam{Coup}}
    \tadOperation{ajouter}{\tadDeuxParams{Coups}{Coup}}{\tadUnParam{Coups}}
    \tadOperation{estPrésent}{\tadDeuxParams{Coups}{Coup}}{\tadUnParam{Booléen}}
    \tadOperation{nbCoups}{\tadUnParam{Coups}{}}{\tadUnParam{Naturel}}
\end{tadOperations}

\begin{tadSemantiques}{Coups}
    \tadSemantique{estCoupValide}{L’opération qui permet de vérifier qu'un coup est valide, c'est à dire s'il n'a pas été déjà fait et si sa position est conforme par rapports à celles d'autres pions y compris les pions adversaires.}
    \tadSemantique{iemeCoup}{L'opération qui permet d'obtenir l'ième coup.}
    \tadSemantique{ajouter}{L'opération qui permet d'ajouter un nouveau coup dans l'ensemble des coups.}
    \tadSemantique{estPrésent}{L'opération qui permet de vérifier si un coup a été déjà fait.}
    \tadSemantique{nbCoups}{L'opération qui permet de compter combien des coups on été faits en total.}
\end{tadSemantiques}
\begin{tadPreconditions}
    \tadPrecondition{iemeCoup(cps,i)} \textit{i $\leq$ nbCoups(cps)}
\end{tadPreconditions}
\begin{tadAxiomes}
    \tadAxiome{estCoupValable(cps,coup(p,pos))} \textit{ non estPrésent(cps,coup)} et \textit{estValide(coup)}
    \tadAxiome{ajouter(cps,coup)} \textit{nbCoups(cps) = nbCoups(cps) + 1}
    \tadAxiome{nbCoups(cps()) = 0}
    \tadAxiome{iemeCoup(ajouter(cps,coup),i) = coup}
\end{tadAxiomes}
\end{tad}
\newpage
